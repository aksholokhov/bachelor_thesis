\section{Demand-Response Overview}
Smart Grids drew a lot of attention in the recent years. Traditionally, the term "grid" denotes an electricity system, that supports electricity generation, transmission, distribution and control. Most of them use for direct energy delivery from several large generators to consumers. In contrast, a Smart Grid is an electricity grid which utilise two-way flows of energy and information to establish automated and distributed next-generation energy delivery network \cite{Fang2012}. These intelligent technologies are incorporated across the entire system which improve its efficiency, safety and reliability \cite{Gao2012}. 

One of the distinguishing features of such type of networks is demand manageability. This concept of the Demand-Side Management (DSM) includes all activities aiming to alter the consumer's demand profile to make it match the supply and to effectively incorporate renewable energy sources \cite{Alizadeh2012}. Nowadays the major activity in DSM is Demand-Response (DR), which is considered as a subset of the broader category of DSM, together with energy efficiency and conservation programs \cite{Palensky2011}. According to the United States Department of energy, Demand-Response is "a tariff or program established to motivate changes in electric use by end-use customers, in response to changes in the price of electricity over time, or to give incentive payments designed to induce lower electricity use at times of high market prices or when grid reliability is jeopardized" \cite{DepartmentofEnergyUSA2006}. According to \cite{Vardakas2015}, the main objectives of the application of DR scheme may be summarised as:
\begin{itemize}
    \item Reduction of the total energy consumption both on demand and transmission sides.
    \item Reduction of the total needed power generation in order to eliminate the need of activating expensive-to-run power plants to meet peak demands. Such overall consumption curtailment may help governments and energy providers to meet their pollution obligations \cite{DepartmentofEnergyUSA2006, Shishlov2016, UnitedNations/FrameworkConventiononClimateChange2015}.
    \item Efficient incorporation of renewable energy sources through making the demand follow the available supply fluctuations. Such incorporation may significantly increase the overall system's reliability in regions with high penetration of wind farms and solar panels \cite{Santacana2010}. 
    \item Reduction or even elimination of overloads in distribution systems.
\end{itemize} 

As shown in figure \missing{insert scheme} the principal DR-scheme consists of cooperation of four main participants: a) an Aggregator, b) a System Operator (SO), c) Power Generation Unit(s) d) and Power Consumer(s) \cite{Medina2010}. Their interaction is a cyclic process typically started by the SO, which determines the preferred power consumption and sends it to the Aggregator. Next the Aggregator chooses participating loads from available, calculates possible change in demand and sends it back to the SO. And finally the Operator informs the most available substations about the upcoming demand. In such scheme the Aggregator provides the grid's intelligence executing optimisation procedures pr revealing problems in distribution system \cite{Vardakas2015}. 

DR Schemes distinct in control architecture they utilise, and in motivation to participate which they provide to customers.

\paragraph{DR Schemes by its architecture} DR schemes may be classified into centralised and distributed programs \cite{Zhou2012}, according to where the decision for the execution program are made. In centralised schemes load activations are managed only by the central utility. Such schemes are easy to implement, but turn to be hard-headed in large and complex systems. However it remains an effective approach for controlling ensembles of termostatically controlled loads \cite{Hao2015}, charging systems for electro vehicles \cite{Yano2012} and commercial consumers \cite{Motegi2007}. For example \missing{Add example}


\paragraph{DR Schemes by its motivation}
The proposed motivation schemes usually adopt either price-based or incentive-based approach. 



