\section{Demand-Response Overview}


One of the distinguishing features of Smart Grid networks is demand manageability. This concept of the Demand-Side Management (DSM) includes all activities aiming to alter the consumer's demand profile to make it match the supply or to effectively incorporate renewable energy sources \cite{Alizadeh2012}. Nowadays the major activity in DSM is Demand-Response (DR)\cite{Palensky2011}. According to the United States Department of energy, Demand-Response is "a tariff or program established to motivate changes in electric use by end-use customers, in response to changes in the price of electricity over time, or to give incentive payments designed to induce lower electricity use at times of high market prices or when grid reliability is jeopardized" \cite{DepartmentofEnergyUSA2006}. As was summarised in \cite{Vardakas2015}, the main objectives of the application of DR are:
\begin{itemize}
    \item Reduction of the total energy consumption both on demand and transmission sides. Such overall consumption curtailment may help governments and energy providers to meet their pollution obligations \cite{DepartmentofEnergyUSA2006, Shishlov2016, UnitedNations/FrameworkConventiononClimateChange2015}.
    \item Reduction of the maximal needed power generation in order to eliminate the need of activating expensive-to-run power plants to meet peak demands. 
    \item Efficient incorporation of renewable energy sources through making the demand follow the available local supply fluctuations. Such incorporation may significantly increase the overall system's reliability in regions with high penetration of wind farms and solar panels \cite{Santacana2010}. 
    \item Reduction or even elimination of overloads in distribution systems shifting peak demands' time for a subset of consumers.
\end{itemize} 

Amount of DR service providers (also know as Aggregators) in US energy market soared up recent years due to emerging new intelligent solution and overall market liberalisation \todo{is it true?} \todo{Add citation}. A typical aggregator company represents several thousand households or a few dozens of commercial consumers (e.g. downtown office buildings). \todo{some statistics} One of the most typical aggregator's business models is providing the curtailment of its customers' demand according to city's energy system operator's requests.      

As shown in figure \todo{insert RD scheme} the principal DR-scheme consists of cooperation of four main participants: a) an Aggregator, b) a System Operator (SO), c) Power Generation Unit(s) d) and Power Consumer(s) \cite{Medina2010}. Their interaction is a cyclic process typically started by the SO, which determines the preferred power consumption and sends it to the Aggregator. Next the Aggregator chooses participating loads from available, calculates possible change in demand and sends it back to the SO. And finally the Operator informs the most available substations about the upcoming demand. In such scheme the Aggregator provides the grid's intelligence executing optimisation procedures pr revealing problems in distribution system \cite{Vardakas2015}. 

DR Schemes distinct in control architecture they utilise, and in motivation to participate which they provide to customers \todo{classification scheme}.

\paragraph{DR Schemes by its architecture} DR schemes may be classified into centralised and distributed programs \cite{Zhou2012}, according to where the decision for the execution program are made. In centralised schemes load activations are managed only by the central utility. Such schemes are easy to implement, but turn to be hard-headed in large and complex systems. However it remains an effective approach for controlling ensembles of termostatically controlled loads \cite{Hao2015}, charging systems for electro vehicles \cite{Yano2012} and commercial consumers \cite{Motegi2007}. For example \todo{Example of centrallized and effective}. In contrast, in distributed architectures all consumers are interconnected, which enables them to make individual decisions on system reliability and so the neediness to curtail the consumption. \cite{Fan2011}. As this architecture does not require a master node, it turns to be more scalable and reliable, ensuring users personal data protection as well. 

\paragraph{DR Schemes by its motivation}
Most of proposed motivation schemes usually adopt either price-based or incentive-based approach \cite{Vardakas2015}. The former achieves altering consumers behaviour through adjusting the energy price \cite{Aghaei2013}. It assumes that consumers are sensitive to energy price, so they will decrease their demand as energy price increases and vice versa. \textit{Time-of-Use}\cite{Aghaei2013},  \textit{Critical Peak Pricing}\cite{Zhou2012} and \textit{Real-Time Pricing} \cite{Chen2011} -- are examples of successful implementation of this approach. In comparison, the incentive-based approach assumes incentive payments to participating customers. These payments may depend on the amount of demand curtailment, or may not. In some schemes participants may reject the curtailment request and hence to forgo the payment (e.g. EDRP \cite{Aalami2010}), while in another schemes participating is mandatory, however an aggregator is limited to the total amount of such requests (typically 125 hours / year) \cite{Chen2013}. 



