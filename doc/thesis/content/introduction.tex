\chapter{Introduction}
Energy efficiency turns to be the one of main challenge for governments industries across the planet. Lack of resources together with rocketed energy demand in most countries urge peoples to increase efficiency of their energy systems. One of the most promising technologies in this field is Smart Grid. Traditionally, the term "grid" denotes an electricity system, that supports energy generation, transmission, distribution and control. From the beginning of XX century majority of them use for direct energy transmission from several large generators to consumers. In contrast, a Smart Grid is an electricity grid which utilise two-way flows of energy and information to establish automated and distributed next-generation energy delivery network \cite{Fang2012}. Being incorporated across the entire system these intelligent technologies improve overall system's efficiency, safety and reliability \cite{Gao2012}. 

The main advantage of Smart Grid systems is a potential for altering end-user consumption behaviour. In particular, it allows to shift peak demands to stabilise daily consumption profile. Such interaction between energy operators and consumers is called Demand-Response (henceforth DR). Plenty of DR schemes have been proposed recent years, and some of them are in use in US energy market. They differ in what they assume about the system, what type of grid architecture they utilise, what type of motivation for end users they propose, and what mathematical approach to emerge system's intelligence they apply. The main challenge in designing of DR schemes is dealing with a lack of communication between server and customers, which in the real life prohibits to address each device individually. However, most studies assume two-sided information stream, which leads to significant infrastructure investment therefore detaining proliferation of such intelligent systems into the market. 

In this study we propose a novel scheme which is aimed to minimise communication requirements without a lessening in functionality of adjusting grid consumption. In particular, it requires to know only aggregated grid's energy consumption, which is significantly more realistic than two-sided communication architecture. To pursue this we consider the set of consumers as an ensemble of devices, which behaviour is representable by finite Markov chains. This is an extension of \cite{Chertkov2017} ideas, who proves this approach to be viable and efficient. To tackle optimisation subroutines we consider this problem as a Linear Contextual Bandit With Knapsack (BwK \cite{Badanidiyuru2013}) setup, applying the algorithms proposed in \cite{Agrawal2015}. The main contribution of this work is proposed Markov-Сhains-based ensemble model, which provides context for contextual bandit learning algorithms and has a naturally linear dependency of reward over this context. It enables to apply advanced bandit algorithms like \cite{Agrawal2015} to Demand-Response problems without loss of its convergence guarantees and making no additional assumptions. 

\section{Demand-Response Overview}


One of the distinguishing features of Smart Grid networks is demand manageability. This concept of the Demand-Side Management (DSM) includes all activities aiming to alter the consumer's demand profile to make it match the supply or to effectively incorporate renewable energy sources \cite{Alizadeh2012}. Nowadays the major activity in DSM is Demand-Response (DR)\cite{Palensky2011}. According to the United States Department of energy, Demand-Response is "a tariff or program established to motivate changes in electric use by end-use customers, in response to changes in the price of electricity over time, or to give incentive payments designed to induce lower electricity use at times of high market prices or when grid reliability is jeopardized" \cite{DepartmentofEnergyUSA2006}. As was summarised in \cite{Vardakas2015}, the main objectives of the application of DR are:
\begin{itemize}
    \item Reduction of the total energy consumption both on demand and transmission sides. Such overall consumption curtailment may help governments and energy providers to meet their pollution obligations \cite{DepartmentofEnergyUSA2006, Shishlov2016, UnitedNations/FrameworkConventiononClimateChange2015}.
    \item Reduction of the maximal needed power generation in order to eliminate the need of activating expensive-to-run power plants to meet peak demands. 
    \item Efficient incorporation of renewable energy sources through making the demand follow the available local supply fluctuations. Such incorporation may significantly increase the overall system's reliability in regions with high penetration of wind farms and solar panels \cite{Santacana2010}. 
    \item Reduction or even elimination of overloads in distribution systems shifting peak demands' time for a subset of consumers.
\end{itemize} 

Amount of DR service providers (also know as Aggregators) in US energy market soared up recent years due to emerging new intelligent solution and overall market liberalisation \unsure{is it true?} \todo{Add citation}. A typical aggregator company represents several thousand households or a few dozens of commercial consumers (e.g. downtown office buildings). \missing{some statistics} One of the most typical aggregator's business models is providing   

As shown in figure \missing{insert scheme} the principal DR-scheme consists of cooperation of four main participants: a) an Aggregator, b) a System Operator (SO), c) Power Generation Unit(s) d) and Power Consumer(s) \cite{Medina2010}. Their interaction is a cyclic process typically started by the SO, which determines the preferred power consumption and sends it to the Aggregator. Next the Aggregator chooses participating loads from available, calculates possible change in demand and sends it back to the SO. And finally the Operator informs the most available substations about the upcoming demand. In such scheme the Aggregator provides the grid's intelligence executing optimisation procedures pr revealing problems in distribution system \cite{Vardakas2015}. 

DR Schemes distinct in control architecture they utilise, and in motivation to participate which they provide to customers.

\paragraph{DR Schemes by its architecture} DR schemes may be classified into centralised and distributed programs \cite{Zhou2012}, according to where the decision for the execution program are made. In centralised schemes load activations are managed only by the central utility. Such schemes are easy to implement, but turn to be hard-headed in large and complex systems. However it remains an effective approach for controlling ensembles of termostatically controlled loads \cite{Hao2015}, charging systems for electro vehicles \cite{Yano2012} and commercial consumers \cite{Motegi2007}. For example \missing{Add example}


\paragraph{DR Schemes by its motivation}
The proposed motivation schemes usually adopt either price-based or incentive-based approach. 





\section{Multiarmed Bandits}
According to \cite{Szepesvari2010}: 

\begin{definition}Reinforcement Learning is a learning paradigm concerned with learning to control a complex system so to maximize a numerical performance measure that express some long-term objective.    
\end{definition}

 The amount of papers devoted to Reinforcement Learning has soared up recent years due to significant advances in various domains, such as robotics, medical trials, adversarial games and recommendation systems.

 The most typical setting where reinforcement learning operates is an iterative process of agent-environment interactions. Formally, let us use the following notation. Define the set of possible agent's actions $a$ as $\A = \{a_1, \dots, a_m \}$, the reward $r \in \R_{+}$, the state of the environment $Q \in \Q$, the time $t \in [0, 1 \dots, T] \subseteq \N$ and the context vector as some vector $x \in \X \subseteq R^k$. In addition, let the policy $$\nu(Q, x, t):\Q \times \X \times \N \to \A$$ be some function that maps current knowledge on the environment to the set of actions. At the moment $t$, after taking into consideration the current context $x_t$ and state of the environment $Q_t$ the agent performs an action $a_t$ according to his current policy $\nu_t$. Next, the environment samples a reward $r_t$ and a context $x_t$ from some unknown distribution $\pi(x,r|a)$, and change its state from $Q_t$ to $Q_{t+1}$ according some unknown Markov Decision Rule $M:\, Q \times Q \to [0, 1]$. (see fig \ref{fig:MAB_scheme}). Finally, the agent use gained knowledge to adjusts its policy $\nu_t$. The long-term aim of the agent is to minimise the regret ${\cal R}^\nu_T$ defined as the difference between performances under the best policy available and under some policy $\nu$:
 
\[
        {\cal R}^\nu_T = \sum_{t=1}^T(r^*(t) - r^{\nu}(t))
\]
 
\begin{figure}
    \centering
        \tikz[scale=3]{
        \begin{scope}[nodes={draw, ultra thick}]
            \node (agent) [rounded rectangle] {Agent};
            \node (environment) [right = of agent, rounded rectangle] {Environment};
        \end{scope}
        \path[->]
            (agent) edge [bend left] node [above] {$a_t$} (environment)
            (environment) edge [bend left] node [below] {$r_t, \, x_{t+1}, Q_{t+1}$} (agent); 
        }
    \label{fig:MAB_scheme}
        \caption{Agent-environment interaction}
\end{figure}
 
  Multi-armed bandit problem can be defined as a Reinforcement Learning setup with a fixed state of the environment. If the context $x$ is absent, then such setup is called "context-free MAB", and "contextual MAB" otherwise. 
  
  \subsection{Context-free Multiarmed Bandits}
  Variety of papers published in three recent decades is devoted to the general context-free multi armed bandit setup. One of the first algorithms -- $\varepsilon$-greedy -- was proposed by \cite{Auer2002}: it pulls an arbitrary arm with a probability of $\varepsilon$ and the best arm according to the current policy otherwise. This is due to the exploration-exploitation balance problem: without this arbitrary steps the agent may fail to gain enough statistics for determining the best policy. However, $\varepsilon$ is a hyperparameter which can't be automatically tuned. Another approach here is so-called "optimism in the face of uncertainty" \cite{Lai1985} according to which the learner should choose an arm with the best Upper Confidence Bound estimation of the reward. A well known algorithm which has implemented this technique is UCB1 \cite{Auer2002}, later analysed and improved by \cite{Audibert2009}. The later not only often outperforms UCB1, but also had shown to be essentially unimprovable under the assumption that the variances of the reward associated with some of the actions are small.
  
  \subsection{Contextual Multiarmed Bandits}
  When the context is available, the agent may take it into consideration while learning the optimal policy. The finite-context case (when the set of possible contexts is finite) was examined by \cite{Auer2002b}, who proposed to consider these set of context as a set $\Q$ of environment states and then apply known reinforcement learning algorithms, such EXP3. 
  
  In contrast to the regular bandit problem, the contextual bandit setup tends to be tougher problem to solve. One approach is to assume the particular dependency model between the arms' context $x_{a,t}$ and the expected reward $r_t$. So, the next important class of contextual bandits is so-called linear contextual bandits, which setup accepts the following assumption:
  
  \begin{definition}
    \label{def:linear_relisability_assumption}
      Linear Realisability Assumption: there is a unknown vector $\theta^* \in \R^k$, such that: 
        \begin{equation*}
                r_{t,a} = \E\left[r_{t,a}|x_{t,a}\right] + \varepsilon_t, \quad s.t. \,\E\left[r_{t,a}|x_{t,a}\right] = x_{t,a}^T\theta^*,\, \varepsilon \sim \N(0, \sigma^2)
        \end{equation*}  
        where $x_{t,a}$ is a context vector for arm $a$.
  \end{definition}
  There are at least two distinguish approaches to algorithms design for this problem class. The first one is an adaptation of UCB1 approach: LinUCB \cite{Li2010}, which tackles optimisation subroutine on $\theta$ with a ridge regression. Later, \cite{Abbasi-Yadkori2011} provided better theoretical analysis for this algorithm by eliminating the assumption that the reward is identically distributed over time and arms, which is usually far from true. The similar case with different regularisation strategy was examined by \cite{Auer2003}. Another approach is to adapt Thompson Sapling technique \cite{Li2011} for balancing exploration and exploitation to the linear reward model. It was done by \cite{Agrawal2013}. This particular linear contextual bandit setup will be widely used later in this study. For further information on contextual bandit problems one is recommended to examine \cite{Zhou2015}.
  
  \subsection{Multiarmed Bandits with Knapsack}
  In real life, while learning, the agent may consume one or more supply of globally constrained resources, limits on which are usually known beforehand. For example, in experimental medical trials possibility of particular treatment depends on the total amount of meds available. Another example is showing advertisements which is constrained not only by the number of users but by the client's advertisement budget as well. The first systematisation and regret bounds for general BwK setup \cite{Badanidiyuru2013} appeared within last five years, the first generalisation to contextualised case -- in 2015: \cite{Agrawal2015}.
  
  According to \cite{Badanidiyuru2013}, the formal problem setup for this case is the following: 
  
  \begin{definition} \textbf{BwK Setup:} over a sequence of steps the agent choose an action $a_t$ and observes not only the reward $r_t$ but also a resource consumption vector $d \in R^d$, where $i$-th component $d^i$ represents a consumption of $i$-th resource. For each resource $d^i$ there is a pre-specified budget $B_i$, known for the agent. The process stops at the moment $T$ when for some $i$ $\sum_{t=0}^Td^i_t > B_i$ or when the time horizon is reached. The goal of the agent is to minimise the regret $\RR_T$ as the distance to accumulated reward $OPT$ under the best policy available.     
  \end{definition}

  
  Note that in this setup the time horizon $T$ is considered as another resource with a budget $T$ and a unit consumption on each step independently on actions being performed.
  
  The Bandit with Knapsack (BwK) problem turns to be significantly more challenging than the conventional MAB setup as the agent needs to solve the stochastic multidimensional version of the Knapsack problem. Such limitation dramatically affects the optimal strategy for the agent, which is more complex in comparison to the "best-arm" strategy for classical MAB. In particular, \cite{Papadimitriou1999} showed that even with knowledge on all latent distribution the problem of revealing the best policy remains at least P-SPACE hard. Hence, the regret is generally uncomputable in such problems as we do not know the best policy $OPT$. 
  
  Two algorithms for the most general BwK setup was proposed by \cite{Badanidiyuru2013}: \textit{PrimalDualBwK} and \textit{BalancedExploration}. Both of them utilise upper confidence bounds design principles and experience suboptimal regret. The BwK adaptation of Thompson Sampling technique was introduced by \cite{Xia2015}, and the case of "convex knapsacks" was examined by \cite{Agrawal2014}. The linear contextual BwK setup (linCBwK) was investigated by \cite{Agrawal2015}. As we will use this setup later in this study, let us consider it in details.
  
  \begin{definition}
    \label{def:lincbwk_setup}
   \textbf{linCBwK setup \cite{Agrawal2015}}:
      There are $m$ "arms" $\{a_i\}_{i=1}^m$. Let $B$ be the budget. In each round the agent observes context vectors $x_{a,t} \in \R^k$ for each arm $a$, chooses some action $a_t$ and observes a reward $r_t \in [0, 1]$ and a consumption vector $d_t \in [0,1]^d$. Also the decision maker may always choose a "no-action" option in order to get 0 reward and \textbf{0} consumption. In each round the a tuple $\{x_{a,t}, r_t, d_t\}$ is sampled from unknown distribution $D$ is independent on everything before. Assume that there exist an unknown vector $\theta^* \in [0,1]^k$ and a matrix $W^* \in [0,1]^{k \times d}$ such that for all arms $a$ given the context $x_{a,t}$ and history $H_{t-1}$,
      \[
        \E\left[r_t|a, x_{t,a}, H_{t-1}\right] = x_{a,t}^T\theta^*, \quad \E\left[d_{t}|x_{a,t}, H_{t-1} \right]
      \]
      Define the regret $\RR_T$ in $T$ rounds as 
      \[
        \RR_T := OPT - \sum_{t=1}^T r_t
      \]
      where OPT is the expected reward under the best static policy avaliable. 
      
      The goal of the agent in linCBwK is to minimise the regret or, equally, to maximise the reward. 
      
  \end{definition} 

The learning algorithm (henceforth libCBwK-learning) for this case was proposed by \cite{Agrawal2015}. It is used high confidence estimations for unknown parameters $\theta^*$ and $W^*$ (so called "confidence ellipsoids") which is a well known and widely used approach in developing RL algorithms (e.g. see \cite{Auer2003}). It was also proven to have near-optimal regret bounds:

\begin{theorem} (Agrawal 2015) If $B > k^{1/2}T^{3/4}$ then with probability at least $1-\delta$ 
\[
    \RR_T = O\left((\frac{OPT}{B} + 1)k\sqrt{T\log(dT/\delta)\log(T)}\right)
\]       
\end{theorem}

One may see linCBwK-learning algorithm's pseudocode in \cite{Agrawal2015}.
\section{Reinforcement Learning in Demand-Response}