\chapter{Conclusion}
In this study were examined the problem of controlling the energy consumption of an ensemble of loads. In particular, we worked with a tough case when the only feedback available is the ensemble's aggregated consumption. We applied two recent algorithms to this problem: PrimalDualBwK and linCBwK. We embedded a Markov model of the ensemble to the second algorithm to enable the application of contextual bandits solvers. Moreover, it allowed to avoid the use of additional data, such the number of loads accepting request. Both of the algorithms were tested on synthetic data, where linCBwK had demonstrated lower regret and more conscious budget expenditures. This is a good example when, being properly interconnected, a physical model and a machine learning algorithm may collaborate for achieving better performance.

\paragraph{Future works} The next important step is to conduct an experiment on real devices. Also, it is crucial to eliminate the assumption that "all days are the same", treating unknown variables as non-stationery time-series over time. 